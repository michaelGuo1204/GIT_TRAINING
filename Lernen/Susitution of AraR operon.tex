\documentclass[a4paper,10pt,twocolumn]{article}
\input{structure.tex}
\usepackage{subfigure}
\title{Subsitution of AraR operon}
\date{\today}
\begin{document}
    \thispagestyle{firstpage}
    \maketitle
    \lettrineabstract{Since weeks have we stuck in the search of the mathematical model of AraR and AraE promoter in Bacillus $subtilis$,
    but apparently, no progress has been made because a minor field it is to research arabinose operon in Bacillus $subtilis$.
    Hence we need to come up with a alternative choice:subsitute the AraR with a negative regulatory operon in E.$coli$.}
    \section{Similarity of LacI and AraR}
    As we have learnt in molecular biology,LacI promotion sysytem is a negative regulation sysytem.As the concentration of lactose rises ,LacI will binds to 
    arabinose and then release from $lacI$,relieve the cell form the negative regulation.People call this status \textbf{induced} state.\citep{Kuhlman2007}

    Lac is not a mono protein,it is a protein in consists of two asymmetric subunits,which will be transcripted and translated in dependent of lacZY,which are the 
    structural gene of lactose operon.The presence of lactose will change the formation of two LacI subunits,which lead to a loss of interaction between protein 
    and DNA.Such a contraption will enable LacI to response to the change of lactose level.
    \begin{figure}
        \includegraphics[width=\linewidth]{Picture/lac.png} % Figure image
        \caption{Summary of Lac operon} % Figure caption
        \label{lac} % Label for referencing with \ref{bear}
    \end{figure}
    
    It's also discovered that AraR in $Bacillus subtilis$ functions in a same pattern as LacI ,it is also a nagetive regulatory operon.The raise of arabinose
    concentration will lead to the release of AraR from promoter AraE.The mechanism of AraR to regulate the expression is very close to what will happend on LacI.
    \begin{figure}
        \includegraphics[width=\linewidth]{Picture/ara.png} % Figure image
        \caption{Summary of Ara operon} % Figure caption
        \label{ara} % Label for referencing with \ref{bear}
    \end{figure}

    Apart from theoretic summary,these two operon also present many similarity in experimental data.Taking the expression data into consideration \ref{compare}
    \begin{figure}
        \subfigure[ara]{
            \begin{minipage}[t]{0.5\linewidth}
            \centering
            \includegraphics[width=1in]{Picture/ara-induced.png}
            %\caption{fig1}
            \end{minipage}%
            }%
            \subfigure[lac]{
            \begin{minipage}[t]{0.5\linewidth}
            \centering
            \includegraphics[width=1in]{Picture/lac-induces.png}
            %\caption{fig2}
            \end{minipage}%
            }%
    \caption{Comparism of ara and lac operon}
    \label{compare}
    \end{figure}
    We can observe a similar trend in both ara and lac operon model.The expression of their structure gene just increased.Therefore,i come up with this idea to subsitute ara with lac.
    
    But,there is one distinct difference between these two operons.
    The AraR is a \textbf{monomer} protein.There does not exist any interaction within it.We will make a further discussion afterwards.

    \section{Modeling of Lac promoter}
    
\end{document}